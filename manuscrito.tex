\documentclass[11pt,]{article}
\usepackage[left=1in,top=1in,right=1in,bottom=1in]{geometry}
\newcommand*{\authorfont}{\fontfamily{phv}\selectfont}
\usepackage[]{mathpazo}


  \usepackage[T1]{fontenc}
  \usepackage[utf8]{inputenc}



\usepackage{abstract}
\renewcommand{\abstractname}{}    % clear the title
\renewcommand{\absnamepos}{empty} % originally center

\renewenvironment{abstract}
 {{%
    \setlength{\leftmargin}{0mm}
    \setlength{\rightmargin}{\leftmargin}%
  }%
  \relax}
 {\endlist}

\makeatletter
\def\@maketitle{%
  \newpage
%  \null
%  \vskip 2em%
%  \begin{center}%
  \let \footnote \thanks
    {\fontsize{18}{20}\selectfont\raggedright  \setlength{\parindent}{0pt} \@title \par}%
}
%\fi
\makeatother




\setcounter{secnumdepth}{3}



\title{Título\\
Subtítulo\\
Subtítulo  }



\author{\Large Darihana Linares Laureano\vspace{0.05in} \newline\normalsize\emph{Estudiante de Licenciatura en Geografía mención recursos naturales y
ecoturismo, Universidad Autónoma de Santo Domingo (UASD)}  }


\date{}

\usepackage{titlesec}

\titleformat*{\section}{\normalsize\bfseries}
\titleformat*{\subsection}{\normalsize\itshape}
\titleformat*{\subsubsection}{\normalsize\itshape}
\titleformat*{\paragraph}{\normalsize\itshape}
\titleformat*{\subparagraph}{\normalsize\itshape}

\titlespacing{\section}
{0pt}{36pt}{0pt}
\titlespacing{\subsection}
{0pt}{36pt}{0pt}
\titlespacing{\subsubsection}
{0pt}{36pt}{0pt}





\newtheorem{hypothesis}{Hypothesis}
\usepackage{setspace}

\makeatletter
\@ifpackageloaded{hyperref}{}{%
\ifxetex
  \PassOptionsToPackage{hyphens}{url}\usepackage[setpagesize=false, % page size defined by xetex
              unicode=false, % unicode breaks when used with xetex
              xetex]{hyperref}
\else
  \PassOptionsToPackage{hyphens}{url}\usepackage[unicode=true]{hyperref}
\fi
}

\@ifpackageloaded{color}{
    \PassOptionsToPackage{usenames,dvipsnames}{color}
}{%
    \usepackage[usenames,dvipsnames]{color}
}
\makeatother
\hypersetup{breaklinks=true,
            bookmarks=true,
            pdfauthor={Darihana Linares Laureano (Estudiante de Licenciatura en Geografía mención recursos naturales y
ecoturismo, Universidad Autónoma de Santo Domingo (UASD))},
             pdfkeywords = {palabra clave 1, palabra clave 2},  
            pdftitle={Título\\
Subtítulo\\
Subtítulo},
            colorlinks=true,
            citecolor=blue,
            urlcolor=blue,
            linkcolor=magenta,
            pdfborder={0 0 0}}
\urlstyle{same}  % don't use monospace font for urls

% set default figure placement to htbp
\makeatletter
\def\fps@figure{htbp}
\makeatother

\usepackage{pdflscape} \newcommand{\blandscape}{\begin{landscape}}
\newcommand{\elandscape}{\end{landscape}} \usepackage{float}
\floatplacement{figure}{H}
\newcommand{\beginsupplement}{ \setcounter{table}{0} \renewcommand{\thetable}{S\arabic{table}} \setcounter{figure}{0} \renewcommand{\thefigure}{S\arabic{figure}} }


% add tightlist ----------
\providecommand{\tightlist}{%
\setlength{\itemsep}{0pt}\setlength{\parskip}{0pt}}

\begin{document}
	
% \pagenumbering{arabic}% resets `page` counter to 1 
%
% \maketitle

{% \usefont{T1}{pnc}{m}{n}
\setlength{\parindent}{0pt}
\thispagestyle{plain}
{\fontsize{18}{20}\selectfont\raggedright 
\maketitle  % title \par  

}

{
   \vskip 13.5pt\relax \normalsize\fontsize{11}{12} 
\textbf{\authorfont Darihana Linares Laureano} \hskip 15pt \emph{\small Estudiante de Licenciatura en Geografía mención recursos naturales y
ecoturismo, Universidad Autónoma de Santo Domingo (UASD)}   

}

}








\begin{abstract}

    \hbox{\vrule height .2pt width 39.14pc}

    \vskip 8.5pt % \small 

\noindent Resumen del manuscrito


\vskip 8.5pt \noindent \emph{Keywords}: palabra clave 1, palabra clave 2 \par

    \hbox{\vrule height .2pt width 39.14pc}



\end{abstract}


\vskip 6.5pt


\noindent  \section{Introducción}\label{introducciuxf3n}

Desde mediados del siglo XVII es posible visualizar el interés del
hombre a estudiar la flora, la fauna y el medio en el que están en
conjunto con las interacciones que se producen entre ellos, pero no es
hasta mediados del siglo XIX cuando se introduce el término Ecología y
su definición, que se empieza a englobar en este tipo de estudios en una
categoria (De la Llata Loyola, 2003). A partir de este punto se
reconocieron distintos campos de estudios y se implementaron nuevos
métodos de análisis, entre los cuales destacan la ecología númerica y
métodos como el análisis multivariático. Según P. Legendre \& Legendre
(2012), la ecología númerica no es más que una de las disciplinas de la
ecología cuantitativa, la cual a la vez es una de las divisiones de la
ecología matemática.

La ecología númerica se concentra en el estudio y análisis de conjuntos
de datos ecológicos, a fin de poder detallar y comprender la
configuración de los conjuntos de datos, combinando diversas
perspectivas númericas y disciplinas, procedentes de la Geografía, las
matemáticas física, taxonomía númerica, párametros estadísticos y otros
más (P. Legendre \& Legendre, 2012). El análisis de conjuntos de datos
ecológicos, especialmente de la flora, resulta importante tanto para la
ciencia, como para la economía o el sector de salud, por eso se han
establecido distintas parcelas permantes de medición y monitoreo
forestal donde se colectan datos sobre la diversidad forestal, su
estructura, el crecimiento y su productividad (Pineda, 2014). En el
estudio de las plantas a través de la Ecología Númerica se usan diversas
técnicas que permiten obtener información sobre el rango de asociación,
agrupamiento, ordenamiento, diversidad, autocorrelación, etcétera,
diferentes herramientas para usar las ténicas (ver tabla
\ref{tab:tecnicas}).

La medición de la asociación es un coeficiente que sirve para medir y
asociar los datos de variables cualitativos y cuantitativos. La medición
de estas variables se puede hacer por dos modos el Q y el R, el primero
consiste en hacer una comparación de un duo de objetos y el segundo
consiste en realizar una descripción de un par de objetos y luego
compararlos. Para el modo Q se miden la asociación según la similaridad
o disimilaridad de un par de objetos. Mientras que para el modo R se
mide el grado de dependecia existente entre las variables, entre los
cuales se puede mencionar la covarianza o el coeficiente de correlación
(Borcard, Gillet, Legendre, \& others, 2011).

En cambio, el análisis de agrupamiento o cluster análisis es una técnica
que consiste en separar un conjunto de datos y luego estructurarlos, sin
dejar uno fuera de lugar, como subconjuntos con distintas categorias o
jerarquias de acuerdo a sus característica. La finalidad de hacer un
agrupamiento es identificar los pequeños grupos dispersos en un espacio
discreto pero constante, este agrupamiento divide en conjunto de objetos
a estudiar, por lo que es necesario e importante que cada objeto
agrupado en otros subgrupos no se encuentre en otros (Borcard et al.,
2011).

En el caso del análisis de ordenaminto, son técnicas que consisten en
simplificar la magnitud de los datos. Todas estas técnicas muestran las
predisposiciones esenciales de variabilidad de cada dato que se
encuentran en un campo de dimensiones simplificadas, organizando los
ejes con rangos decreciente de varianza explicada en cada uno de los
ejes sucesivo, de forma convencional. Estos tipos de análisis pueden ser
tanto no restringido o simple, como restringido o canónico. En donde
para el primero, las tendencias o predisposiciones del grupo que
interesa no esta restringida por otro grupo. Entre sus técnicas
principales de análisis estan los de componentes principales (PCA)
basado en un vector propio y se utiliza en datos cuantitativos sin
tratamiento preservando la distancia euclídea, los de correspondencia
(CA) que se usa en datos frecuentes con dimensiones uniformes y
positivos, y los de coordenadas principales (PCoA) que se concentra en
organizar las matrices de disimilaridad, usualmente, con el modo Q en
vez de tablas de sitio por variables (Borcard et al., 2011).

Mientras que, para el segundo análisis de ordenamiento, restringido o
canónico, a diferencia de la simple, es una técnica que relaciona dos o
más conjunto de datos en el proceso de organización u ordenación.
Algunas de sus técnicas principales son análisis de redundancia (RDA)
que consiste en la combinación de la regresión y PCA que funciona como
una extensión de diversos análisis que muestran la respuesta
multivariante de datos, y el análisis de correspondencia canónica (CCA)
que funciona como un aproximado de una regresión Gaussiana
multivariante, además, este se caracteriza por organizar las especies en
todos los ejes canónicos acore a su configuración ecológica óptima
(Borcard et al., 2011).

En cuanto a la diversidad, según Borcard et al. (2011) y Magurran
(1988), esto alude a la variedad y cantidad de especies en un espacio
determinado, esta variedad también se produce a nivel de comunidad. La
diversidad puede La diversidad va desde la diversidad local, hasta
heterogeneidad espacial de esta diversidad. La diversidad de especies
considerada como un número unico puede ser medida por la riqueza o la
rarificación de especies usando la notación q o por la presencia o
ausencia de los datos de esta. Según Whittaker (1972), el entendimiento
de la diversidad y los cambios que conllevan, asociados a la
configuración del relieve los componentes alfa, beta y gamma serian de
mucha utilidad. Donde la primera se refiere a la riqueza de las especies
existentes en una comunidad determinada, considerada homogénea. La
segunda, se trata de el rango de cambios que se producen en la
estructura de especies que estan en diferentes comunidades en un
espacio. Y la última, se refiere a la riqueza de especies de forma
conjuntiva que hay en una comunidad y que integra un espacio
determinado.

La autocorrelación, según Borcard et al. (2011), forma parte de los
análisis espaciales aplicados a datos ecológicos, que se produce por
distintos procesos y que mide puntos cercanos para afirmar si estos
poseen valores similares o distintos, por lo que la correlación puede
ser una correlación positiva o negativa.

La Myrtaceae son una familia de plantas de árboles y arbustos bastante
numerosas compuestas caracterizadas por ser leñosas. Esta familia de
plantas pertenece al orden de los Myrtales, teniendo a nivel mundial 129
géneros y aproximadamente 5330 especies (Perez \& R, s.f.). De sus
caracterisiticas físicas se destacan sus hojas simples y opuestas,
tienden a ser perennes por lo que resulta poco común ver individuos
caducifolios, sus flores son hermafroditas y comunmente son de color
blanco con simetría radial, según la especie de esta familia los frutos
tienen forma de bayas o cápsulas secas. De esta familia el género más
conocido es el Eucalypto Eucalyptus por sus propiedades medicinales y su
madera dura. Esta familia esta distribuida en todos los continentes,
pero predominan en América, África y Oceanía, en climas templados,
tropicales y subtropicales. Muchos consideran que la importancia de esta
familia está en lo económico, como la producción de frutas para venta de
zumos y mermeladas, producción de madera, producción de papel, de
carbón, además de ser usadas en la industria farmacéutica, se usa en la
cosmetología y en la producción de especias (Almeida, 2019; Britannica,
2016; Lorenzo-Cáceres, s.f.).

El estudio de la biodiversidad de la flora en especial de una especie
vegetal es importante, debido a que permite conocer sus caracteristicas
propias,ecosistemicas, su distribución, su capacidad productiva, su
potencial de uso y aportes a los humanos como a su ecosistema. Por todo
lo anterior, es importante determinar el objetivo de este estudio es: a)
identificar si los grupos de mi familia, Myrtaceae, se organizan de
forma discontinua y acorde a la composición de las especies; b) indagar
si existe algun tipo de patrón que sea o no sea consistente con alguna
variable ambiental o atributo; c) determinar cuantas especies
indicadoras hay o si hay alguna con preferencia por ciertas condiciones
ambientales o atributos; d) investigar si hay tendencias de ordenación
que se distingan

\section{Área de estudio}\label{uxe1rea-de-estudio}

BCI (Isla de Barrio Colorado) es una isla que pertenece a Panamá, con
una parcela permanente de medición y monitoreo forestal con

\subsection{Pregutas de investigación a
incorporar.}\label{pregutas-de-investigaciuxf3n-a-incorporar.}

Técnicas de ordenación (ordination analysis, to): En un espacio
bidimensional, ¿existen tendencias apreciables de ordenación de las
especies de mi familia seleccionada? Si existen tendencias de
ordenación, ¿se asocian éstas con variables ambientales/atributos?

Diversidad (di): Según los análisis de estimación de riqueza, ¿está
suficientemente representada mi familia? Consideremos como buena
representación un 85\% ¿Existe asociación de la diversidad alpha con
variables ambientales/atributos? ¿Con cuáles? ¿Existe contribución local
o por alguna especie a la diversidad beta?

Ecología espacial (ee): ¿Alguna(s) especies de mi familia presenta(n)
patrón aglomerado? ¿Cuál(es)? ¿Se asocia con alguna variable? ¿Predicen
bien la ocurrencia de dicha(s) especie(s) los modelos de distribución de
especies (SDM)?

\section{Metodología}\label{metodologuxeda}

\ldots

\section{Resultados}\label{resultados}

Ver tabla \ref{tab:abun_sp} y figura \ref{fig:abun_sp_q}

\section{Discusión}\label{discusiuxf3n}

\section{Agradecimientos}\label{agradecimientos}

\section{Información de soporte}\label{informaciuxf3n-de-soporte}

\ldots

\section{\texorpdfstring{\emph{Script}
reproducible}{Script reproducible}}\label{script-reproducible}

\ldots

\section*{Referencias}\label{referencias}
\addcontentsline{toc}{section}{Referencias}

\hypertarget{refs}{}
\hypertarget{ref-sandra2019myrtaceae}{}
Almeida, S. (2019). \emph{Myrtaceae, familia}. Retrieved from
\url{https://knoow.net/es/ciencias-tierra-vida/biologia-es/myrtaceae-familia/}

\hypertarget{ref-borcard2011numerical}{}
Borcard, D., Gillet, F., Legendre, P., \& others. (2011).
\emph{Numerical ecology with r} (Vol. 2). Springer.

\hypertarget{ref-encymyrtaceae}{}
Britannica, E. (2016). \emph{Myrtaceae}. Retrieved from
\url{https://www.britannica.com/plant/Myrtaceae}

\hypertarget{ref-de2003ecologia}{}
De la Llata Loyola, M. D. (2003). \emph{Ecología y medio ambiente}.
Editorial Progreso.

\hypertarget{ref-legendre2012numerical}{}
Legendre, P., \& Legendre, L. (2012). \emph{Numerical ecology}.
Elsevier.

\hypertarget{ref-josemyrtaceae}{}
Lorenzo-Cáceres, J. M. S. de. (s.f.). \emph{Familia myrtaceae}.
Retrieved from \url{https://www.arbolesornamentales.es/Myrtaceae.htm}

\hypertarget{ref-magurran1988ecological}{}
Magurran, A. E. (1988). \emph{Ecological diversity and its measurement}.
Princeton university press.

\hypertarget{ref-pereztree}{}
Perez, R., \& R, C. (s.f.). \emph{Tree atlas of panama}. Retrieved from
\url{http://ctfs.si.edu/PanamaAtlas/famdescr.php?Family=Myrtaceae}

\hypertarget{ref-pineda2014analisis}{}
Pineda, P. (2014). \emph{Análisis del sistema de parcelas permanentes de
medición en los bosques de guatemala. informe final}. Guatemala:
Proyecto" Sistemas de información sobre la productividad de
los~\ldots{}.

\hypertarget{ref-whittaker1972evolution}{}
Whittaker, R. H. (1972). Evolution and measurement of species diversity.
\emph{Taxon}, \emph{21}(2-3), 213--251.




\newpage
\singlespacing 
\end{document}
